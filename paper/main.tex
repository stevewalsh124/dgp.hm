\documentclass[11pt]{article}
\usepackage{hyperref}
\usepackage{amsmath, amsfonts, amssymb, mathrsfs}
\usepackage{dcolumn}
\usepackage{caption}
\usepackage{subcaption}
\usepackage{filemod}
\usepackage{natbib}
\usepackage[ruled, vlined]{algorithm2e}
\usepackage{floatrow}
\usepackage{setspace}
\usepackage{verbatim}
\usepackage{graphicx}

\hypersetup{
    colorlinks=true,
    linkcolor=blue, 
    urlcolor=black,
    citecolor=blue, 
    }

\oddsidemargin=0.25in
\evensidemargin=0.25in
\textwidth=7in
\textheight=8.75in
\topmargin=-.5in
\addtolength{\oddsidemargin}{-.5in}
\addtolength{\evensidemargin}{-.5in}
\footskip=0.5in
%\doublespacing

\title{Deep Gaussian Processes and Hierarchical Modeling with Functional Data: A Case Study in Cosmological Power Spectra}
\author{Stephen A. Walsh\thanks{Corresponding author: Division of Natural Sciences, 
        Math and Technology, Elms College, {\tt walshst@elms.edu}} \and 
        Annie S. Booth\thanks{Department of Statistics, North Carolina State University} \and
        David Higdon\thanks{Department of Statistics, Virginia Tech} \and
        Marco A.R. Ferreira\footnotemark[3]}
\date{\today}


\begin{document}

\maketitle
\bigskip

\begin{abstract} 
Understanding the structure of our universe and how matter is spread out and expanding is an area of active research. As cosmological surveys grow in complexity, developing surrogate models to efficiently predict matter power spectra is an important research area in cosmology. In this work, we synthesize methods from deep Gaussian processes, Bayesian hierarchical modeling, and basis functions in a novel way to estimate and predict matter power spectra (which are functional in nature) for different cosmologies. Our method has favorable results compared to the benchmark cosmological emulator (Cosmic Emu), and our code is available in a repository.
\end{abstract}

%\vfill

\section{Introduction}
%%%%%%%%%%%%%%%%%%%%%%%%%%%%%%%%%%%%%%%%%%%%%%%%%%%%%%%%%%%%%%%%%%%%%%%%%%%%%%%

To further our understanding of the structure and movement of the universe, researchers utilize numerous cosmological surveys such as the Sloan Digital Sky Survey \citep{york2000sloan} and the upcoming Nancy Grace Roman Space Telescope \citep{Dore2019WFIRST} which are continuously growing in complexity. With these tools, cosmologists aim to continue learning more about cosmic acceleration \citep{caldwell2009physics}. To this end, one important ingredient to aid in this understanding of our universe is the matter power spectrum. 

The matter power spectrum is a fundamental concept in cosmology and describes the distribution of matter as a function of spatial scale. It is often represented as a function of wavenumber $k$ (units Mpc$^{-1}$), which is inversely related to spatial scale. On large scales (i.e., small $k$ values), cosmic expansion dynamics and the matter power spectrum behave according to linear perturbation theory \citep{pietroni2008flowing, lesgourgues2009non}. On smaller scales, non-linear dynamics require the use of computationally intensive simulations; the Coyote Universe \citep{lawrence2010coyote} and more recently the Mira-Titan Universe \citep{moran2023mira} are two simulation suites dedicated to this effort. 

Leveraging these simulations allows for emulation and prediction of the matter power spectrum under varying specifications for eight different cosmological parameters (i.e., different cosmologies). Estimating and understanding the influence of these parameters on cosmic expansion are areas of active research. In addition to presenting the Mira-Titan simulation suite, \cite{moran2023mira} builds off of previous spectrum emulation \citep{lawrence2017mira} to provide the final emulator (Cosmic Emu) based on the full suite of simulations.

The Cosmic Emu emulator is constructed by (process convolutions, ...). Segue to our method. 

Due to the nonstationary behavior expected to be seen in power spectra (e.g., from baryonic acoustic oscillations), we expand on the previous emulation methods and propose a novel synthesis of methods to quantify uncertainty and predict matter power spectra for different cosmologies using deep Gaussian processes (DGPs). In order to handle multiple realizations of functional output (from the Mira-Titan simulation suite), we use hierarchical and basis functions modeling within our DGP framework to accurately estimate and predict power spectra for different cosmologies. When comparing to Cosmic Emu, we see favorable results for our method. The code to reproduce these results are available at \url{https://github.com/stevewalsh124/dgp.hm}.

\section{Data}

Here, we provide further details on the form of the data (in the introduction, we only briefly introduce it).  Include a figure of the raw data for a particular cosmology.  We describe the different modeling components and the regions of the space where they are trusted.  We also describe how each cosmology is a simulation based on 8 inputs, and the output $\mathcal{P}(k)$ is functional.

We will handle the data in two stages.  First, we will focus on an individual cosmology and obtain the best predicted curve.  Second, we will use these curves to predict the curve for a held-out cosmology.  These two stages form the main two sections of this paper.

Images:
\begin{itemize}
    \item Example cosmology showing 16 low res, 1 high res, 1 pert, wt avg, and where each output is deemed unbiased
\end{itemize}

\section{Bayesian Hierarchical Modeling for Particular Cosmologies}

The main purpose of this section is to describe our first contribution: the use of a Bayesian hierarchical model to estimate the true curve (and provide UQ) for a particular cosmology.  We focus solely on one cosmology, independent of the others.

\subsection{Gaussian Process Model}

Here, we introduce the Bayesian GP model.  We mention the tapering and the formulation of the block covariance matrix.  As well as the posterior conditioning that provides the closed-form solution for the posterior mean and variance.  We can cite Steve's dissertation.

\subsection{Deep Gaussian Process Model}

Here, we expand on the previous section by incorporating a latent layer warping \citep{damianou2013deep}.  We discuss the elliptical slice sampling of $W$, but otherwise it is plugged in to the methods of the previous section with very few changes.

\subsection{Simulation Study}

Here, we can provide results from a sim study comparing the GP and DGP, in which case the true curve is known.

\subsection{Actual results (but more aptly named)}

Estimate one cosmology with 50,000 runs, and use this as a starting point for each of the others
Here we can provide figures of the predicted mean and variances for a subset of the actual cosmologies.

Images:
\begin{itemize}
    \item Illustration of DGP fit (post mean and UQ, could show cosmicEMU as well?)
    \item estimated warping (or, how all 111 warpings compare)
\end{itemize}


\section{Prediction for new cosmologies}

This section details our second contribution - the use of the Bayesian DGP model of the previous section to predict the curve for held-out cosmologies.

\subsection{The model}

Here we describe the PCA-GP model.  We discuss how we use the predicted curves that we got previously as training data.

\begin{itemize}
    \item show breakdown of estimated mean trend, basis functions/PCs, and weights of a particular PC/BF.
\end{itemize}

\subsection{Actual results (but more aptly named)}

Here we show results on held-out cosmologies.  We compare to cosmicEMU and Jared's convolution approach. We can also discuss estimation of main effects, interactions amongst the 8 parameters, and variance decomposition.

\begin{itemize}
    \item compare the 4 with boxplot: our DGP training, DGP testing, vs proc conv training (aka Jared's files), proc conv testing (aka Cosmic EMU)
    \item compare the 4's RMSEs as a function of $k$
    \item Main effects
\end{itemize}

\section{Discussion}

Brief recap, directions for future research: BSS-ANOVA in lieu of DGPs, hydrodynamical sims or TCs as other applications of this methodology. A more detailed sensitivity analysis for the parameters, modeling $W$ across different cosmologies.

\section{Appendix}

The different appendices from dissertation as necessary. Can also put details of dataset in supplementary material.

\bibliographystyle{jasa}
\bibliography{references}

\end{document}
